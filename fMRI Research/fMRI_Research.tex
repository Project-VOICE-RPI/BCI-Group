\documentclass[12pt, research paper]{report}
\usepackage{graphicx}
\title{Project VOICE: Annotated Bibliography}
\author{Sharon Lin}
\date{Fall 2024}

\begin{document}
	
	\maketitle
	
	\section*{Introduction}
	
	\noindent\textbf{Project Description:} Today, non-verbal children are believed to have minimal intelligence and cognitive abilities. We are aiming to prove this false. The Voice Project initiative aims to reverse biases in academia and show that communication is still possible without words or signs. Our current work consists of integrating computer software with biomedical engineering and neurology to explore non-verbal and non-symbolic communication through brain monitoring applications.
	
	\vspace{10pt}
	
	\section*{Annotated Bibliography} 
	\textbf{Title:} Overview of Functional Magnetic Resonance Imaging
	
	\noindent \textbf{Author(s):} Gary H. Glover
	\bigskip 
	
	Blood Oxygen Level Dependent (BOLD) functional magnetic resonance imaging (fMRI) is a technique that tracks brain activity by detecting changes un deoxyhaemoglobin concentration, which reflect alterations in neural metabolism during tasks or at rest. Introduced in 1990, fMRI has become essential for cognitive neuroscience research, clinical applications such as surgical planning, and evaluating treatment outcomes. Its widespread use is due to its non-invasive approach, relatively low cost, good spatial resolution, and availability on standard clinical MRI scanners. Over the years, advancements in fMRI technology have addressed challenges like low contrast-to-noise ratios, image distortion, and signal dropout. The field has now shifted toward more sophisticated uses, including pattern classification and statistical modelling to infer complex cognitive brain states. Emerging application involve using fMRI as a biomarker for diagnosing diseases, monitoring therapy, studying pharmacological effects, and enhancing behavioural interventions. Additionally, real-time fMRI feedback is being explored for therapeutic uses, such as pain reduction and mood regulation. As the understanding of brain networks continues to evolve, fMRI is poised to play a crucial role in advancing neuroscience and clinical purposes.
	\bigskip
	
	\noindent \textbf{Significance:} BOLD fMRI relates directly to The VOICE Project initiative's goals by providing a powerful tool for demonstrating the cognitive abilities of non-verbal children. As fMRI measures brain activity linked to neural metabolism, it can reveal patterns of thought and cognitive processing that are otherwise invisible, challenging the misconception that non-verbal individuals have limited intelligence. By integrating computer software, biomedical engineering, and neurology, our project can utilise fMRI data to develop real-time brain monitoring applications, enabling non-verbal children to communicate without relying on traditional language or symbols. 
	\bigskip

	\noindent Glover GH. Overview of functional magnetic resonance imaging. Neurosurg Clin N Am. 2011 Apr;22(2):133-9, vii. doi: 10.1016/j.nec.2010.11.001. PMID: 21435566; PMCID: PMC3073717.
	\bigskip
	\bigskip
	
	\noindent \textbf{Title:} Endogenous Oscillations and Networks in Functional Magnetic Resonance Imaging
	
	\noindent \textbf{Author(s):} Peter A. Bandettini, Ed Bullmore
	\bigskip
	
	Since the early 1990s, functional magnetic resonance imaging (fMRI) has become a dominant tool for mapping human brain function, surpassing positron emission tomography (PET) due to its high resolution, sensitivity, and ease of use. While early fMRI research focused on identifying brain regions activated during specific tasks, recent developments have expanded on the field's scope to explore the brain's endogenous dynamics, such as resting-state networks and spontaneous low-frequency oscillations that reveal spatially coherent functional connections. This shift marks a transition from reflexive models of brain function, which emphasise externally driven responses, to intrinsic models that consider the brain's self-sustained oscillatory activity. Emerging research combines fMRI with other modalities like EEG to cross-validate findings, enhancing our understanding of endogenous brain activity. Key challenges remain in analysing and interpreting this data, particularly in distinguishing neuronal from non-neuronal sources of fluctuations. Moreover, the potential clinical applications of fMRI are being explored, including using endogenous markers to characterise neuro-psychiatric disorders, evaluate treatment effects, and understand normal brain organisation. This evolving focus on intrinsic brain dynamics signals a significant period of growth in fMRI research, with implications for both basic neuroscience and clinical practice.
	\bigskip 
	
	\noindent \textbf{Significance:} This is relevant to the project because it provides a foundation for understanding the brain's intrinsic activity, crucial for revealing cognitive capabilities in non-verbal children. The shift in fMRI research towards studying the brain's self-sustained activity aligns with the project's goal of exploring non-verbal communication through brain monitoring. This approach offers an alternative to traditional communication assessments that require verbal responses, enabling the detection of functional connections in the brain. Additionally, combining fMRI with modalities like EEG supports the project's multidisciplinary approach, enhancing the validation of neural signs associated with communication. 
	\bigskip 
	
	\noindent Bandettini, P. A., \& Bullmore, E. (2008). Endogenous oscillations and networks in functional magnetic resonance imaging. Human brain mapping, 29(7), 737–739. https://doi.org/10.1002/hbm.20607
	\bigskip 
	\bigskip
	
	\noindent \textbf{Title:} Functional fMRI Experiments: acquisition, analysis, and interpretation of data
	
	\noindent \textbf{Author(s):} N.F. Ramsey, H. Hoogduin, J.M.Jansma
	\bigskip 
	
	The paper reviews the current use and future development of functional MRI (fMRI) in neuroscience. It covers key aspects like data acquisition, analysis, and interpretation, focusing on improving sensitivity to brain activity changes and designing effective tasks. fMRI primarily measures changes in blood oxygen levels, with ongoing advancements expected to enhance accuracy. Data analysis involves reducing noise and modeling brain activity responses, while the design of tasks should align with specific research goals. Future research aims for more sophisticated approaches to understanding brain functions.
	\bigskip
	
	\noindent \textbf{Significance:} The insights from fMRI research align with my project's goals of aiding nonverbal children in communication by providing a deeper understanding of brain activity patterns associated with cognitive functions and communication. The paper emphasizes improving sensitivity to brain function-related changes and accurately modeling involved brain functions, which are crucial when developing a device for nonverbal children.	It could benefit from fMRI techniques, such as measuring blood oxygen level-dependent (BOLD) signals, to identify brain regions activated during nonverbal communication or cognitive tasks. Understanding these activation patterns could guide the design of tasks or stimuli to enhance the device's effectiveness in interpreting the users' intentions. Additionally, advanced analysis techniques mentioned in the paper, like connectivity analysis, could help extract meaningful signals that can be translated into communication outputs, making the device more accurate and user-friendly.
	\bigskip
	
	\noindent \textbf{Title:} Full activation pattern mapping by simultaneous deep brain stimulation and fMRI with graphene fiber electrodes
	
	\noindent \textbf{Author(s):} Siyuan Zhao, Gen Li, Chuanjun Tong, Wenjing Chen, Puxin Wang, Jiankun Dai, Xuefeng Fu, Zheng Xu, Xiaojun Liu, Linlin Lu, Zhifeng Liang, Xiaojie Duan
	\bigskip 

	This study evaluates graphene fiber (GF) microelectrodes as an alternative to traditional metal electrodes for deep brain stimulation (DBS) in Parkinson’s disease (PD) models. Metal electrodes, like platinum-iridium, often interfere with MRI, causing distortions that limit brain response mapping. GF microelectrodes, by contrast, reduce these MRI artifacts, enabling clearer imaging near the stimulation site. The researchers fabricated GF electrodes with high conductivity and coated them with Parylene-C, achieving a charge-injection capacity significantly higher than platinum-iridium. GF electrodes produced minimal MRI interference, allowing for unbiased visualization of brain regions affected by stimulation. Testing in PD rats demonstrated the GF electrodes’ effectiveness in reducing motor symptoms like bradykinesia while providing stable, low-impedance stimulation over time. These results suggest that GF electrodes may offer a more precise and artifact-free alternative for DBS in neurological research and treatment.
	\bigskip 
	
	\noindent \textbf{Significance:} The significance of this research is directly tied to the project’s ultimate goal of developing a communication device for nonverbal children. By understanding and advancing BCI technology, we can gain insights into how brain signals can be accurately captured, interpreted, and translated into alternative forms of communication. This aligns with our objective to create a device that allows nonverbal children to convey thoughts, emotions, or needs through brain activity instead of speech, thus opening new, accessible communication pathways for individuals who lack verbal ability.

	
	
	
	
\end{document}