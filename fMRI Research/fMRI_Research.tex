\documentclass[12pt, research paper]{report}
\usepackage{graphicx}
\title{Project VOICE: Annotated Bibliography}
\author{Sharon Lin}
\date{Fall 2024}

\begin{document}
	
	\maketitle
	
	\section*{Introduction}
	
	\noindent\textbf{Project Description:} Today, non-verbal children are believed to have minimal intelligence and cognitive abilities. We are aiming to prove this false. The Voice Project initiative aims to reverse biases in academia and show that communication is still possible without words or signs. Our current work consists of integrating computer software with biomedical engineering and neurology to explore non-verbal and non-symbolic communication through brain monitoring applications.
	
	\vspace{10pt}
	
	\section*{Annotated Bibliography} 
	\textbf{Title:} Overview of Functional Magnetic Resonance Imaging
	
	\noindent \textbf{Author(s):} Gary H. Glover
	\bigskip 
	
	Blood Oxygen Level Dependent (BOLD) functional magnetic resonance imaging (fMRI) is a technique that tracks brain activity by detecting changes un deoxyhaemoglobin concentration, which reflect alterations in neural metabolism during tasks or at rest. Introduced in 1990, fMRI has become essential for cognitive neuroscience research, clinical applications such as surgical planning, and evaluating treatment outcomes. Its widespread use is due to its non-invasive approach, relatively low cost, good spatial resolution, and availability on standard clinical MRI scanners. Over the years, advancements in fMRI technology have addressed challenges like low contrast-to-noise ratios, image distortion, and signal dropout. The field has now shifted toward more sophisticated uses, including pattern classification and statistical modelling to infer complex cognitive brain states. Emerging application involve using fMRI as a biomarker for diagnosing diseases, monitoring therapy, studying pharmacological effects, and enhancing behavioural interventions. Additionally, real-time fMRI feedback is being explored for therapeutic uses, such as pain reduction and mood regulation. As the understanding of brain networks continues to evolve, fMRI is poised to play a crucial role in advancing neuroscience and clinical purposes.
	\bigskip
	
	\noindent \textbf{Significance:} BOLD fMRI relates directly to The VOICE Project initiative's goals by providing a powerful tool for demonstrating the cognitive abilities of non-verbal children. As fMRI measures brain activity linked to neural metabolism, it can reveal patterns of thought and cognitive processing that are otherwise invisible, challenging the misconception that non-verbal individuals have limited intelligence. By integrating computer software, biomedical engineering, and neurology, our project can utilise fMRI data to develop real-time brain monitoring applications, enabling non-verbal children to communicate without relying on traditional language or symbols. 
	\bigskip
	
	\noindent Glover GH. Overview of functional magnetic resonance imaging. Neurosurg Clin N Am. 2011 Apr;22(2):133-9, vii. doi: 10.1016/j.nec.2010.11.001. PMID: 21435566; PMCID: PMC3073717.
	\bigskip
	
	\noindent \textbf{Title:} Endogenous Oscillations and Networks in Functional Magnetic Resonance Imaging
	
	\noindent \textbf{Author(s):} Peter A. Bandettini, Ed Bullmore
	\bigskip
	
	Since the early 1990s, functional magnetic resonance imaging (fMRI) has become a dominant tool for mapping human brain function, surpassing positron emission tomography (PET) due to its high resolution, sensitivity, and ease of use. While early fMRI research focused on identifying brain regions activated during specific tasks, recent developments have expanded on the field's scope to explore the brain's endogenous dynamics, such as resting-state networks and spontaneous low-frequency oscillations that reveal spatially coherent functional connections. This shift marks a transition from reflexive models of brain function, which emphasise externally driven responses, to intrinsic models that consider the brain's self-sustained oscillatory activity. Emerging research combines fMRI with other modalities like EEG to cross-validate findings, enhancing our understanding of endogenous brain activity. Key challenges remain in analysing and interpreting this data, particularly in distinguishing neuronal from non-neuronal sources of fluctuations. Moreover, the potential clinical applications of fMRI are being explored, including using endogenous markers to characterise neuro-psychiatric disorders, evaluate treatment effects, and understand normal brain organisation. This evolving focus on intrinsic brain dynamics signals a significant period of growth in fMRI research, with implications for both basic neuroscience and clinical practice.
	\bigskip 
	
	\noindent \textbf{Significance:} This is relevant to the project because it provides a foundation for understanding the brain's intrinsic activity, crucial for revealing cognitive capabilities in non-verbal children. The shift in fMRI research towards studying the brain's self-sustained activity aligns with the project's goal of exploring non-verbal communication through brain monitoring. This approach offers an alternative to traditional communication assessments that require verbal responses, enabling the detection of functional connections in the brain. Additionally, combining fMRI with modalities like EEG supports the project's multidisciplinary approach, enhancing the validation of neural signs associated with communication. 
	\bigskip 
	
	\noindent Bandettini, P. A., \& Bullmore, E. (2008). Endogenous oscillations and networks in functional magnetic resonance imaging. Human brain mapping, 29(7), 737–739. https://doi.org/10.1002/hbm.20607
	\bigskip 
	
	\noindent \textbf{Title:} Current Methods and New Directions in Resting State fMRI
	
	\noindent \textbf{Author(s):} Jackie Yang, Suril Gohel, Behrose Vaccha
	\bigskip 
	
	
\end{document}