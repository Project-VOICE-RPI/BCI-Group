\documentclass{article}
\usepackage{graphicx} % Required for inserting images
\usepackage{hyperref} % Required for hyperlinks

\title{Research Papers and Studies}
\author{Tahira Tariq}
\date{October 11th, 2024}

\begin{document}

\maketitle

\noindent\textbf{Topic: How brain waves (beta/gamma) associated with mirror neuron activity and neuroplasticity reflect nonverbal communication cues like body language and facial expressions.}

\subsubsection*{The Mirror-Neuron System and the Consequences of Its Dysfunction}
\textit{Authors: Marco Iacoboni, Mirella Dapretto}

This paper delves into the mirror neuron system, particularly how it helps humans understand the actions and intentions of others, which is crucial for interpreting nonverbal cues like body language and facial expressions. It also explores the potential dysfunction of this system in conditions like autism,\\affecting nonverbal communication.

Iacoboni, M., \& Dapretto, M. (2006). The mirror neuron system and the consequences of its dysfunction. \textit{Nature Reviews Neuroscience}, \textit{7}(12), 942–951. \href{https://doi.org/10.1038/nrn2024}{https://doi.org/10.1038/nrn2024}

\subsubsection*{The Role of Beta-Frequency Neural Oscillations in Motor Control}
\textit{Authors: Nick J. Davis, Simon P. Tomlinson, Helen M. Morgan}

This study discusses how beta oscillations (15-30 Hz) are linked to motor control, particularly how they change during voluntary movement and return to baseline after movement. It explores the hypothesis that beta activity represents the status quo and how its disruption is associated with motor disorders like Parkinson’s disease. The paper also examines the use of transcranial alternating current stimulation (tACS) to investigate the role of these oscillations in\\motor control.

Davis, N.J., Tomlinson, S.P., \& Morgan, H.M. (2012). The Role of\\Beta-Frequency Neural Oscillations in Motor Control. \textit{Journal of Neuroscience}, \textit{32}(2), 403–404. \href{https://doi.org/10.1523/jneurosci.5106-11.2012}{https://doi.org/10.1523/jneurosci.5106-11.2012}

\subsubsection*{Gamma-Band Synchronization in the Macaque Hippocampus and Memory Formation}
\textit{Authors: Michael J. Jutras, Pascal Fries, Elizabeth A. Buffalo}

This paper focuses on gamma-band synchronization (30-100 Hz) in the\\hippocampus and its role in memory formation. It provides evidence that gamma synchronization during the encoding phase of a memory task predicts better subsequent recognition memory. The study highlights the importance of precise timing in neuronal activity for long-term synaptic changes, which are crucial for memory encoding.

Jutras, M.J., Fries, P., \& Buffalo, E.A. (2009). Gamma-Band\\Synchronization in the Macaque Hippocampus and Memory Formation.\\ \textit{Journal of Neuroscience}, \textit{29}(40), 12521–12531.\\\href{https://doi.org/10.1523/jneurosci.0640-09.2009}{https://doi.org/10.1523/jneurosci.0640-09.2009}

\subsubsection*{A Unifying View of the Basis of Social Cognition}
\textit{Authors: Vittorio Gallese, Christian Keysers, Giacomo Rizzolatti}

This paper discusses how mirror neurons are key to understanding not just movements but also emotions through facial expressions and body language. It outlines the relationship between mirror neuron activity and beta/gamma oscillations, contributing to theories about neuroplasticity and learning from social interactions.

Gallese, V., Keysers, C., \& Rizzolatti, G. (2014). A Unifying View of the Basis of Social Cognition. \textit{PhilPapers.org}. \href{http://philpapers.org/rec/GALAUV}{http://philpapers.org/rec/GALAUV}

\subsubsection*{Imaging Functional Neuroplasticity in Human White Matter Tracts}
\textit{Authors: Cathy J. Price, Karl J. Friston}

This paper reviews the use of MRI, particularly diffusion tensor imaging (DTI), to study neuroplasticity in white matter tracts. It discusses how\\motor training can lead to structural and functional changes in white matter, such as the internal capsule and corpus callosum.

Frizzell, T.O., Phull, E., Khan, M., \textit{et al.} (2022). Imaging Functional\\Neuroplasticity in Human White Matter Tracts. \textit{Brain Struct Funct}, \textbf{227}, 381–392. \href{https://doi.org/10.1007/s00429-021-02407-4}{https://doi.org/10.1007/s00429-021-02407-4}

\end{document}