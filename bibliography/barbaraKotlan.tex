\documentclass[12pt]{article}
\usepackage{hyperref}
\usepackage{graphicx} % Required for inserting images

\title{Annotated Bibliography of Current Research}
\author{Barbara Kotlan}
\date{September 2024}

\begin{document}
\maketitle

\section{Paper 1}
\href{https://www.ncbi.nlm.nih.gov/pmc/articles/PMC10385593}{Electroencephalography Signal Processing: A Comprehensive Review and Analysis of Methods and Techniques}
This article provides an extensive overview of various methods and techniques used in processing electroencephalography (EEG) signals. EEG is a non-invasive method for monitoring electrical brain activity through electrodes placed on the scalp, making it useful for studying brain function and diagnosing neurological disorders. This review examines the different stages of EEG signal processing, including data acquisition, pre-processing (artifact removal and noise reduction), feature extraction (identifying relevant signal characteristics), and classification (interpreting and categorizing brain states). The article discusses traditional techniques as well as advanced machine learning approaches, highlighting the challenges and limitations of each method. It also explores applications in brain-computer interfaces, neurorehabilitation, and cognitive state monitoring. The review aims to guide future research and developments in EEG signal processing by offering a detailed analysis of current practices and emerging trends.

\section{Paper 2}
\href{https://jamanetwork.com/journals/jamaneurology/fullarticle/2806244}{Automated Interpretation of Clinical Electroencephalograms Using Artificial Intelligence | Epilepsy and Seizures | JAMA Neurology | JAMA Network}
This article discusses the use of artificial intelligence to improve the interpretation of clinical electroencephalograms, which are used to monitor and diagnose neurological conditions such as epilepsy and seizures. The paper highlights the advantages of AI-driven approaches in enhancing diagnostic accuracy, reducing human error, and increasing the efficiency of EEG analysis in clinical settings. It reviews different AI techniques, including machine learning and deep learning models, for automated EEG interpretation, and discusses their potential applications in detecting epileptic seizures and other neurological abnormalities. The article also considers the limitations and challenges of AI integration into clinical workflows and emphasizes the need for continued research to optimize these technologies for medical use.

\section{Paper 3}
\href{https://www.ncbi.nlm.nih.gov/pmc/articles/PMC9231602/}{Functional Near-Infrared Spectroscopy Reveals Brain Activity on the Move}
This article discusses the use of functional near-infrared spectroscopy (fNIRS) to measure brain activity during movement and real-world tasks. fNIRS is a non-invasive imaging technique that monitors changes in blood oxygenation in the brain, providing insights into neural activity. The paper highlights the advantages of fNIRS for studying brain function in naturalistic settings, as opposed to traditional imaging methods like fMRI, which require subjects to remain still. It reviews the applications of fNIRS in various fields, including neurorehabilitation, cognitive neuroscience, and brain-computer interfaces. The article also addresses the limitations of fNIRS, such as signal depth limitations and motion artifacts, while suggesting potential advancements to enhance its capabilities for mobile brain imaging.

\section{Paper 4}
\href{https://pubs.asha.org/doi/10.1044/2020_AJSLP-19-00050}{Functional Near-Infrared Spectroscopy in the Study of Speech and Language Impairment Across the Life Span: A Systematic Review}
This article focuses on the use of functional near-infrared spectroscopy (fNIRS) in neuroimaging studies involving individuals with speech or language impairments. It highlights the growing role of functional brain imaging in diagnosing and treating communication disorders, especially in cases where traditional methods like fMRI are unsuitable. This systematic review, conducted according to the PRISMA protocol, examined 34 studies using fNIRS across different categories of speech and language impairments, including autism spectrum disorders, developmental language disorders, stuttering, stroke/aphasia, and traumatic brain injury. The findings suggest that fNIRS could offer clinical advantages for early diagnosis, treatment monitoring, and neurofeedback despite some challenges, potentially improving outcomes for individuals with communication impairments.

\section{Paper 5}
\href{https://pmc.ncbi.nlm.nih.gov/articles/PMC9784128/}{A Review of Machine Learning for Near-Infrared Spectroscopy}
This article discusses on the analysis of infrared spectroscopy, with a focus on machine learing algorithms and deep network architectures. The tradtional ML rely on enineer features, while the deep networks use raw features and hidden layers. The preprocessing done by ML is necessary to redue noice from any hardware constraints. The feature selection is also very necessary to minimize computaitonal complexity by solely focusing on the relevant wavelengths. It outlines challenges for future advancments in integrating ML for practical NIR spectroscopy applications. 

\section{Paper 6}
\href{https://www.nature.com/articles/srep36203}{Near-infrared spectroscopy (NIRS)-based eyes-closed brain-computer interface (BCI) using prefrontal cortex activation due to mental arithmetic}
The article focuses on the potential of a near-infrared spectroscopy based BCI that can be operated in a closed eye state. By analyzing changes in hemoglobin levels, researchs we able to comapre mental arithmetic task with baseline to see brain patterns. Closed eye and open eyed both were comparable and accurate, making closed-eye a viable BCI option that allows participants to be more comfortable and focused for the results. 

\section{Paper 7}
\href{https://pubmed.ncbi.nlm.nih.gov/29076307/}{Toward a functional near-infrared spectroscopy-based monitoring of pain assessment for nonverbal patients}
This paper focuses on using NIRS signals with machine learning to classify pain intensity. This was done to see if NIRS signals could aid in diagnosing pain for nonverval patients. To classify the pain intensity, temperature stimuli was tested on 18 total participants. Machine learning used two classifier models, support vector machines and K-nearest neighbor. The KNN model had more accurancy, with 92\% compared to the 91.25\% for SVM. Based on this information, the results suggest that NIRS could be used to help in sucessfully diagnosing pain for nonverbal patients.

\end{document}
