\documentclass[12pt]{article}
\usepackage{hyperref}
\usepackage{graphicx} % Required for inserting images

\title{Annotated Bibliography of Current Research}
\author{Barbara Kotlan}
\date{September 2024}

\begin{document}
\maketitle

\section{Paper 1}
\href{https://www.ncbi.nlm.nih.gov/pmc/articles/PMC10385593}{Electroencephalography Signal Processing: A Comprehensive Review and Analysis of Methods and Techniques}
This article provides an extensive overview of various methods and techniques used in processing electroencephalography (EEG) signals. EEG is a non-invasive method for monitoring electrical brain activity through electrodes placed on the scalp, making it useful for studying brain function and diagnosing neurological disorders. This review examines the different stages of EEG signal processing, including data acquisition, pre-processing (artifact removal and noise reduction), feature extraction (identifying relevant signal characteristics), and classification (interpreting and categorizing brain states). The article discusses traditional techniques as well as advanced machine learning approaches, highlighting the challenges and limitations of each method. It also explores applications in brain-computer interfaces, neurorehabilitation, and cognitive state monitoring. The review aims to guide future research and developments in EEG signal processing by offering a detailed analysis of current practices and emerging trends.

\section{Paper 2}
\href{https://jamanetwork.com/journals/jamaneurology/fullarticle/2806244}{Automated Interpretation of Clinical Electroencephalograms Using Artificial Intelligence | Epilepsy and Seizures | JAMA Neurology | JAMA Network}
This article discusses the use of artificial intelligence to improve the interpretation of clinical electroencephalograms, which are used to monitor and diagnose neurological conditions such as epilepsy and seizures. The paper highlights the advantages of AI-driven approaches in enhancing diagnostic accuracy, reducing human error, and increasing the efficiency of EEG analysis in clinical settings. It reviews different AI techniques, including machine learning and deep learning models, for automated EEG interpretation, and discusses their potential applications in detecting epileptic seizures and other neurological abnormalities. The article also considers the limitations and challenges of AI integration into clinical workflows and emphasizes the need for continued research to optimize these technologies for medical use.

\section{Paper 3}
\href{https://www.ncbi.nlm.nih.gov/pmc/articles/PMC9231602/}{Functional Near-Infrared Spectroscopy Reveals Brain Activity on the Move}
This article discusses the use of functional near-infrared spectroscopy (fNIRS) to measure brain activity during movement and real-world tasks. fNIRS is a non-invasive imaging technique that monitors changes in blood oxygenation in the brain, providing insights into neural activity. The paper highlights the advantages of fNIRS for studying brain function in naturalistic settings, as opposed to traditional imaging methods like fMRI, which require subjects to remain still. It reviews the applications of fNIRS in various fields, including neurorehabilitation, cognitive neuroscience, and brain-computer interfaces. The article also addresses the limitations of fNIRS, such as signal depth limitations and motion artifacts, while suggesting potential advancements to enhance its capabilities for mobile brain imaging.

\end{document}
