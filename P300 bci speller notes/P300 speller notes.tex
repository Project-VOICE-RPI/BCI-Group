\documentclass{article}
\usepackage{url} % For handling URLs
\begin{document}
More info on P300 speller\\
A P300 speller consists of 36 alpha-numeric characters in a 6x6 matrix, the user has to focus their attention on a character, then rows and columns will flash in a random sequence. There are many different variations to the P300 paradigm that try to improve the framework. In the classical variation, a sequence of 12 different flashes (6 rows and 6 columns) is called an iteration. These flashes are an oddball paradigm, so there are two different kinds of stimuli, target and non target, occurring at frequencies of 0.166 and 0.833 respectively. The target response which is more rare should elicit the P300 response, which in this case the target stimulus is the row/column containing the desired character. We need a classification algorithm to distinguish between these stimuli. In ERP-based BCIs, to perform a selection step, many iterations are needed to improve the SNR (signal to noise ratio). Machine learning and optimization can be used to improve the classification performance.\\

Links to EEG data for P300 speller brain-computer interfaces\\
\url{https://www.nature.com/articles/s41597-022-01509-w#ref-CR28} \\
\url{https://www.bbci.de/competition/iii/desc_II.pdf} \\

Bianchi, L., Liti, C., Liuzzi, G. et al. Improving P300 Speller performance by means of optimization and machine learning. Ann Oper Res 312, 1221–1259 (2022). https://doi.org/10.1007/s10479-020-03921-0

\end{document}
