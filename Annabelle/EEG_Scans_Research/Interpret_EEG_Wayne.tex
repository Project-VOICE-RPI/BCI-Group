\documentclass[12pt]{article}

\usepackage{graphicx}

\title{EEG (electroencephalogram)}
\author{Annabelle Chan}
\date{September 2024}

\begin{document}
\maketitle
Link: https://neurology.med.wayne.edu/pdfs/how\_to\_interpret\_and\_eeg\_and\_its\_report.pdf

Author: Marie Atkinson (Wayne university medical school, Comprehensive Epilepsy Program)

Date: July 19, 2010

Managing 
Montages 
Sensitivity - the amplitudes of the wave lines from the scan
The lower the voltage used (in microvolts) the larger of the amplitude, the more you will see 
Usually 7.5 microvolts
Filters
Low Frequency (LF) Filter: 
If you set a frequency at frequency x, eeg will not amplify any frequencies below that number
Usually set at 1.0 Hz
High Frequency (HF) Filter:
If you set a frequency at frequency x, eeg will not amplify any frequencies above that number
Usually set at 35 Hz
Notch Filter:	
Cuts off any activity above and at 60 Hz
The current through plugs is often at 60 Hz and allows people to ignore outside artifacts by other machinery in the room.
How to read 
Background activity
Just based on the frequency (delta, theta, alpha, or beta), we get an overall sense on how the person being tested is doing
Deta 1-3 Hz: Marked slowing
Theta 4-7 Hz: Mildly slow 
Alpha 8-13 Hz: Normal background
Beta >13 Hz: Barbiturates/Benzos (awaken state https://www.ncbi.nlm.nih.gov/books/NBK390343/.)
Usually evaluated in the posterior channel (back part of the complete cerebral cortex) and often in occipital (visual processing area of the brain within the posterior channel).
The person being tested must have their eyes closed.
Symmetry
Asymmetric slowing
Asymmetric slowing is seen with focal lesions, surgery, etc.
Asymmetric slowing is easy to see with A montage
It can be seen by hyperventilation or photic stimulation 
PLEDs (Periodic Lateralized Epileptiform Discharges)
It occurs throughout the entire EEG at a frequency of 1-2 Hz and only in one hemisphere
It is usually seen in acute lesions (stroke, bleed, etc.), postictal, Herpes Encephalitis, CJD, and etc.
Controversial over reading meaning (thought to be inconclusive?)
Stage of alertness
Abnormality
Epileptiform Activity
This classification means the EEG reader saw some abnormalities that is related to seizures (but needs clinical correlation)
These abnormalities include sharp waves, spikes, or slow waves
 Spikes
The duration of the abnormal wave is 70 microvolts or less.
It stands out from the background
On a Bipolar Montage, it needs to have phase reversal to be real
Sharp Wave
The duration is about 70-200 microvolts.
It stands out from the background
On a Bipolar Montage, it needs to have phase reversal to be real
Both sides of the slope should be sloped. If it’s straight on one side, it’s usually an artifact (a noise or disturbance in the data not from the brain).
 Slow Wave
The duration is greater than 200 microvolts.
It stands out from the background.
Actual Seizures
The EEG reader can tell if it’s partial or generalized, status epilepticus, or consistent with primary generalized syndrome 
Seizure should be like a wave, and have a buildup and let down
Triphasic Waves
It usually indicates metabolic or toxic encephalopathy (usually liver failure or renal disease)
Patients always have some degree of encephalopathy (mild to moderate)
It can be seen with Cefepime encephalopathy and Li intoxication 
Usually anterior dominant, diffuse, and bilaterally synchronous 
It has 3 phases to the waveform: negative, positive, negative
It will have a lag anterior to posterior in wave 2 peak
“Looks like a backward check mark” Guru Dr. Shah
Primary Generalized Epilepsy 
Generalized bursts of activity 
Secondary generalized bisynchrony: generalized burst following a localized focal abnormal epileptiform waveform (not primary generalized epilepsy)
Burst Suppression 
When spontaneous, very poor prognosis
Brain Dead/Electrocerebral Silence
Specialized eeg protocol that is not performed very often due to the required time period and artifacts that may be mistaken for brain activity
Breech Rhythm
Seen over a skull defect, namely surgery and has the same frequency as rest of EEG (with a higher amplitude)
Hypsarrhythmia
Seen in infantile spasms (West Syndrome) with a “chaotic” background.
Will have a decremental appearance when child is actually having a spasm 
SSPE (Subacute Sclerosing Panencephalitis)
Encephalitis that tends to affect young boys after experiencing measles illness
Mortality is high and those that survived have intellectual sequelae 
The pattern is different from burst suppression because in between bursts, the background is not suppressed
Artifacts
Eye blink artifact
Only seen in prefrontal channels
If you look at the eye channel, the waveform occurs at the same time just in opposite directions
Muscle artifact 
Appears to be too sharp and usually occurs when patient is agitated or moving
If the HF filter is removed, the results get even worse
EKG artifact 
Corresponds with the QRS complex on the EKG and usually looks like regular spikes transmitted throughout the EEG
Stages of sleep
Often mistaken as abnormal and best seen on bipolar montage in Cz channels
K complexes, Vertex waves, Spindles
Normal EEG
A normal EEG should have alpha frequency background activity, no abnormalities (nothing stands out in the background), no changes in the EEG provoked by photic, hyperventilation, and no asymmetry  
Words in EEG report 
Epileptiform - waveforms are seen that have the potential to cause seizures but clinical correlation still needed
Photic Stimulation - provoking procedure done to induce primary generalized epilepsy or asymmetries 
Driving response - patient reacting normally to the stimulation 
No driving response - no EEG change with photic stimulation 
Hyperventilation - procedure performed to provoke primary generalized epilepsy and provoke asymmetrics. Symmetric physiological slowing is normal 
 Stage 1 sleep - presence of vertex waves and theta slowing during EEG
Stage 2 sleep - presence of K complexes and sleep spindles with delta slowing during EEG
Drowsiness provoke epileptiform activity (the reason why sleep deprived EEG is recommended to truly rule out seizure disorder)
Diffuse slowing - background is theta or delta frequency. It’s seen with encephalopathic state, medication effect, prostical, dementia, and bilateral structural defect
Focal slowing - seen over lesions like tumors, stroke, hippocampal sclerosis, usually indicates structural lesion in area
If the patient is seizing, the report will say seizure or status epilepticus 

\end{document}




