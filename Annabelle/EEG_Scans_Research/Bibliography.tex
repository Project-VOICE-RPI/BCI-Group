\documentclass[12pt]{article}
\usepackage{graphicx} % Required for inserting images

\title{Interperting EEG Scans and regions of the brain that correlate to "yes" or "no"}
\author{Annabelle Chan}
\date{September 2024}

\begin{document}
\maketitle

\section{EEG (electroencephalogram)}
This article provides a general description on EEG scans. It describes the general reasons on why it's done, any risks that come with it, precautions that should be taken, and what's to be expected when an EEG scan is done on you. Reference EEG\_Mayo\_Clinic.tex for more information.

\section{How To Interpret an EEG How To Interpret an EEG and its Report}
This presentation explains how to manage an EEG scan via its sensitivity and filters. It also explains how to read an eeg scans backgrounf activity, interpret abnormality, and identify any artifacts. In addition, it explains the terminology used in an EEG report. Reference Interpret\_EEG\_Wayne.tex for more information.

\section{montages and technicalities}
This article explains how to set up the electrodes for an EEG scan. It explains the benefits and disadvantages for each type of montage. Reference montage.tex for more information.

\section{5 Types Of Brain Waves Frequencies: Gamma, Beta, Alpha, Theta, Delta}
This article explains the 5 types of brain wave frequencies and explains how the presence of them a person's current state. Reference brain\_waves.tex for more information.

\section{Study reveals a universal pattern of brain wave frequencies}
This article shows the pattern of brain frequencies across mammalians for information processing. Reference Brain\_Pattern\_MIT.tex for more information.

\section{UW team links two human brains for question-and-answer experiment}
Researchers at University of Washington run an experiment where two players communicate with each other by sending signals to the brain based on there responses (based on their thoughts alone). Reference washingtonu\_experiment.tex for more information.


\end{document}