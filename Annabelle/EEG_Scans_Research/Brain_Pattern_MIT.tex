\documentclass[12pt]{article}

\title{Universal Pattern of Brain Wave Frequencies}
\author{Annabelle Chan}
\date{October 2024}

\begin{document}
\maketitle
Link: https://news.mit.edu/2024/study-reveals-universal-pattern-brain-wave-frequencies-0118

Author: Anne Trafton

Date: January 18, 2024\newline

Mammalian brain waves are found to be slower in deep cortical layers, while superficial layers generate faster rhythms and that these layers show a distinct pattern of electrical activity in the prefrontal cortex.
The topmost layers’s neuron activity is dominated by rapid oscillations (gamma waves).
The deeper layers have slower oscillations (alpha and beta waves).


According to a study done by Bastos, animals had lower-frequency rhythms in deeper layers that regulated the higher-frequency gamma rhythms in the superficial layers while performing memory tasks.


The brain’s cortex is responsible for thought, planning, and high-level processing of emotion and sensory information with neurons being arranged in 6 layers (each layer having its own distinctive combination of cell types and connections with other brain areas).
Note: Many studies in brain activity reported having difficulty understanding where the activity of neurons originated from within those layers due to each layer being a millimeter thick.


A model from Miller’s lab proposed that the brain’s spatial organization helps it to incorporate new information (with it being carried by high-frequency oscillations into existing memories and brain processes and maintained by low-frequency oscillations)

\end{document}