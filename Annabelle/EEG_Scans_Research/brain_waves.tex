\documentclass{article}
\begin{document}
5 Types Of Brain Waves Frequencies: Gamma, Beta, Alpha, Theta, Delta - MentalHealthDaily
All five type of brains waves will be displayed but one can be more dominant depending on your state of consciousness 
Gamma waves
Involved in higher processing tasks as well as cognitive functioning 
Important for learning, memory, and information processing.
It has a frequency of 40 Hz (which is important for the binding of our senses in regards to perception and involved in learning new material) to 100 Hz. (highest)
People with learning disabilities and those who are mentally challenged tend to have lower gamma activity on average.
Too much means: Anxiety, high arousal, stress
Too little means: ADHD, depression, and learning disabilities 
Optimal for binding senses, cognition, information processing, learning, perception, and REM sleep 
Meditation increases gamma waves
Beta Waves
High frequency and low amplitude brain waves that are usually observed while the person is awake. 
They are involved in conscious thought, logical thinking, and tend to have a stimulating effect.
Correlated to how concentrated and focused we are.
It has a frequency of 12 Hz to 40 Hz (high)
Too much means: Adrenaline, anxiety, high arousal, inability to relax, stress
Too little mean: ADHD, daydreaming, depression, poor cognition
Optimal for conscious focus, memory, and problem solving
Coffee, energy drinks, and various stimulants increase beta waves
Alpha Waves
This frequency range bridges the gap between our conscious thinking and subconscious mind (between theta and beta).
It helps us calm down and promotes feelings of deep relaxation 
If we become too stressed, “alpha blocking” may occur (excessive beta activity and little alpha activity because the beta waves “block” out the production of alpha waves due to high arousal)
It has a frequency of 8 Hz to 12 Hz (moderate)
Too much means: daydreaming, inability to focus, too relaxed
Too little means: Anxiety, high stress, insomnia, and OCD
It is optimal for relaxation
Alcohol, marijuana, relaxants, and some antidepressants increase alpha waves
Theta Waves
This frequency wave is involved in daydreaming and sleep.
Theta waves are correlated to us feeling deep and intense emotions.
It has a frequency range of 4 Hz to 8 Hz (slow)
Too much means: ADHD, depression, hyperactivity, impulsivity, inattentiveness and being in a “highly suggestible” state (due to being in a deeply relaxed, semi-hypnotic state)
Too little means: Anxiety, poor emotional awareness, stress
It is optimal for creativity, emotional connection, intuition, and relaxation
Depressants increase theta waves
Delta waves
They are most often found in infants and young children. As we age, we tend to produce less delta waves even during sleep.
They correlate to the deepest levels of relaxation and restorative, healing sleep and unconscious bodily functions (e.g. heartbeat and digestion)
Adequate production of delta waves helps us feel completely rejuvenated after we wake up from deep sleep.
It has a frequency range of 0 Hz to 4 Hz (slowest).
Too much means: brain injuries, learning problems, inability to think, and severe ADHD
Too little means: inability to rejuvenate body, inability to revitalize the brain, and poor sleep
It is optimal for the functioning of your immune system, natural healing, and restorative/deep sleep.

Study reveals a universal pattern of brain wave frequencies | MIT News | Massachusetts Institute of Technology
Mammalian brain waves are found to be slower in deep cortical layers, while superficial layers generate faster rhythms and that these layers show a distinct pattern of electrical activity in the prefrontal cortex.
The topmost layers’s neuron activity is dominated by rapid oscillations (gamma waves)
The deeper layers have slower oscillations (alpha and beta waves).
According to a study done by Bastos, animals had lower-frequency rhythms in deeper layers that regulated the higher-frequency gamma rhythms in the superficial layers while performing memory tasks.
The brain’s cortex is responsible for thought, planning, and high-level processing of emotion and sensory information with neurons being arranged in 6 layers (each layer having its own distinctive combination of cell types and connections with other brain areas).
Note: Many studies in brain activity reported having difficulty understanding where the activity of neurons originated from within those layers due to each layer being a millimeter thick.
A model from Miller’s lab proposed that the brain’s spatial organization helps it to incorporate new information (with it being carried by high-frequency oscillations into existing memories and brain processes and maintained by low-frequency oscillations)
\end{document}