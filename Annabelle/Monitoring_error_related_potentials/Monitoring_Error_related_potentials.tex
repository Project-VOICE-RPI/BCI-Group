\documentclass[12pt]{article}
\usepackage{graphicx}
\graphicspath{ {./} }

\title{Monitoring Error-related potentials Experiment Notes}
\author{Annabelle Chan}
\date{November 2024}

\begin{document}
\maketitle

\section{Experiment Summary}
6 participants are tasked to identifying whether a cursor on a screen are moving towards the right section on the screen (that will be signified by a colored square). During the experiment, an EEG scan will be ran as the participant accesses the situation.

Each trial lasts approximately 2000 ms and participants have no control over the cursor. Since this experiment is used to monitor erroneous actions, there is about a 0.20 probability for the cursor to move in the wrong direction.

Protocol events (cursor direction and target location) are stored in the EEG recoring. 
\includegraphics[scale=1]{ProtocolEventTable}
In consequence events '5' and '10' mark correct movements, while '6' and '9' mark erroneous movements.

\section{Notes on Data}
The data is in MATLAB variable format. There is experiment data from 6 participants. Each participant had two sessions with each session having 10 runs. Each run has corresponding raw eeg data (data\{i,j\}.eeg) (m samples x n channels) and recording metadata (data\{i,j\}.header). 

The metadata includes the subject number (header.Subject), the session number (header.sample), the recording sampling rate (header.SampleRate), the electrode labels (header.Label), and the recording events (header.EVENT).

\section{Notes on my experiment}
I will be using the recording events and the raw eeg data. Each run will be counted as a data point (120). I will use the recording events to create output data (+1 for correct direction and -1 for incorrect direction), and I will use the raw eeg data as input data. I plan to make two features for the input data.
\end{document}