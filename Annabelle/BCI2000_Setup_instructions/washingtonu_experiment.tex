\documentclass[12pt]{article}
\title{UW team links two human brains for question-and-answer experiment}
\author{Annabelle Chan}
\date{October 2024}
\begin{document}
\maketitle

link: https://www.washington.edu/news/2015/09/23/uw-team-links-two-human-brains-for-question-and-answer-experiment/

date: September 23, 2015

\section{Experiment Summary and Setup}

Researchers at Washington University used direct brain to brain connection to enable pairs of participants to play a question (yes or no) and answer game by transmitting signals from one brain to the other over the Internet.

The respondent wore a cap connected to an EEG machine and is shown an object on a computer screen, and the inquirer sees a list of possible objects and associated questions.

The inquirer sends a question and the respondent answers "yes" or "no" by focusing on one of the two LED lights attachted to the monitor, which flashes at different frequencies. The answer is then sent as a signal to the inquirer's head. The "yes" signal is intense enough to stimulate the visual cortex and cause teh inquirer to see a flash of light (phosphene, a blob, waves, or a thin line created through a brief disruption of the visual field). The goal of the game is to get the inquirer to identify the correct item through this process.

\section{Logistics}

This experiment was carried out in dark rooms in two UW labs located about a mile apart. It had 5 pairs of participants each playing 20 rounds of the game. Each game had 8 objects and 3 questions that would solve the game if answered correctly. The sessions had a random mixture of 10 real games and 10 control games (both structured the same way).

To ensure no outside factors contributed to the final answer, inquirers wore earplugs so they couldn't hear the different sounds produced by the varying stimulation intensities of the "yes" and "no" responses. Additionally, since noise travels through the skull bone, the researchers also changed the stimulation intensities each for "yes" and "no" answers. 

For real games, the researchers repositioned the coil on the inquirer;s head at the start of each game. For contolled games, they added a plastic spacer undetectable to the participant. Both of these measures were to weaken the magnetic field enough to prevent the generation of phosphenes.

Inquirers were not told if they correctly identified the item and none of the participants knew if each game was real or controlled.

72\% of inquirers during the real games guessed the item correctly while on 18\% of them guessed it correctly during the control rounds.


\end{document}